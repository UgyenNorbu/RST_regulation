% Options for packages loaded elsewhere
\PassOptionsToPackage{unicode}{hyperref}
\PassOptionsToPackage{hyphens}{url}
%
\documentclass[
]{book}
\usepackage{lmodern}
\usepackage{amssymb,amsmath}
\usepackage{ifxetex,ifluatex}
\ifnum 0\ifxetex 1\fi\ifluatex 1\fi=0 % if pdftex
  \usepackage[T1]{fontenc}
  \usepackage[utf8]{inputenc}
  \usepackage{textcomp} % provide euro and other symbols
\else % if luatex or xetex
  \usepackage{unicode-math}
  \defaultfontfeatures{Scale=MatchLowercase}
  \defaultfontfeatures[\rmfamily]{Ligatures=TeX,Scale=1}
\fi
% Use upquote if available, for straight quotes in verbatim environments
\IfFileExists{upquote.sty}{\usepackage{upquote}}{}
\IfFileExists{microtype.sty}{% use microtype if available
  \usepackage[]{microtype}
  \UseMicrotypeSet[protrusion]{basicmath} % disable protrusion for tt fonts
}{}
\makeatletter
\@ifundefined{KOMAClassName}{% if non-KOMA class
  \IfFileExists{parskip.sty}{%
    \usepackage{parskip}
  }{% else
    \setlength{\parindent}{0pt}
    \setlength{\parskip}{6pt plus 2pt minus 1pt}}
}{% if KOMA class
  \KOMAoptions{parskip=half}}
\makeatother
\usepackage{xcolor}
\IfFileExists{xurl.sty}{\usepackage{xurl}}{} % add URL line breaks if available
\IfFileExists{bookmark.sty}{\usepackage{bookmark}}{\usepackage{hyperref}}
\hypersetup{
  pdftitle={Road Safety \& Transport Regulation},
  hidelinks,
  pdfcreator={LaTeX via pandoc}}
\urlstyle{same} % disable monospaced font for URLs
\usepackage{longtable,booktabs}
% Correct order of tables after \paragraph or \subparagraph
\usepackage{etoolbox}
\makeatletter
\patchcmd\longtable{\par}{\if@noskipsec\mbox{}\fi\par}{}{}
\makeatother
% Allow footnotes in longtable head/foot
\IfFileExists{footnotehyper.sty}{\usepackage{footnotehyper}}{\usepackage{footnote}}
\makesavenoteenv{longtable}
\usepackage{graphicx,grffile}
\makeatletter
\def\maxwidth{\ifdim\Gin@nat@width>\linewidth\linewidth\else\Gin@nat@width\fi}
\def\maxheight{\ifdim\Gin@nat@height>\textheight\textheight\else\Gin@nat@height\fi}
\makeatother
% Scale images if necessary, so that they will not overflow the page
% margins by default, and it is still possible to overwrite the defaults
% using explicit options in \includegraphics[width, height, ...]{}
\setkeys{Gin}{width=\maxwidth,height=\maxheight,keepaspectratio}
% Set default figure placement to htbp
\makeatletter
\def\fps@figure{htbp}
\makeatother
\setlength{\emergencystretch}{3em} % prevent overfull lines
\providecommand{\tightlist}{%
  \setlength{\itemsep}{0pt}\setlength{\parskip}{0pt}}
\setcounter{secnumdepth}{5}
\usepackage{booktabs}
\usepackage[]{natbib}
\bibliographystyle{apalike}

\title{Road Safety \& Transport Regulation}
\author{}
\date{\vspace{-2.5em}2020-07-02}

\begin{document}
\maketitle

{
\setcounter{tocdepth}{1}
\tableofcontents
}
\hypertarget{purpose-of-the-regulation}{%
\chapter{Purpose of the Regulation}\label{purpose-of-the-regulation}}

The purpose of the regulations is to:

\begin{enumerate}
\def\labelenumi{\alph{enumi}.}
\tightlist
\item
  Set out procedures for motor vehicle registration, renewal, de-registration, re-registration and transfer of ownership, including two wheelers;
\item
  Set out procedures for the granting of learner's license, driver license, including driving instructor's certificate;
\item
  Provide for the fees to be paid for transaction under these regulations;
\item
  Provide powers to the Authority for cancellation or suspension of registration, driving license and instructor certificate for all type of vehicles.
\item
  Provide power to authority to forward any case of impersonation, wilful false declaration, tampering of documents and any other fraudulent practices related to services provided by the authority to the concerned authorities.
\item
  Establish uniform vehicle emission standards throughout the country;
\item
  Encourage use of cleaner fuel and eco-friendly vehicles.
\item
  Promote professionalism and quality training to produce competent drivers
\item
  Provide for monitoring and enforcement mechanism to ensure quality of training provided by the institutes;
\item
  Empower the Authority for retrospective enforcement of traffic rules;
\item
  Set standards for commercial passenger vehicles;
\item
  Provide for certificates to be issued to drivers of and conductors in commercial passenger vehicles;
\item
  Provide for the licensing of commercial passenger vehicles and the conditions of operation;
\item
  Establish systems and procedures for the licensing and operation of taxis in the kingdom;
\item
  Establish vehicle standards; and
\item
  Ensure safe, comfortable and quality services to the users.
\item
  Set of traffic rules for all users of the highway and related areas in the Kingdom of Bhutan.
\item
  Set down the procedures for the use of alcohol testing devices, speed measuring devices, portable weighing devices and other road safety equipment
\item
  Establish the procedures for the issue of Transport Infringement Notices;
\item
  Establish procedures for enforcement of these regulations
\end{enumerate}

\hypertarget{motor-vehicle}{%
\chapter{Motor vehicle}\label{motor-vehicle}}

\hypertarget{category-of-motor-vehicle-for-registration-purpose}{%
\section{Category of motor vehicle for registration purpose}\label{category-of-motor-vehicle-for-registration-purpose}}

For purpose of registration, motor vehicles are categorized as:

\hypertarget{heavy-vehicle}{%
\subsection{Heavy Vehicle}\label{heavy-vehicle}}

\begin{enumerate}
\def\labelenumi{\alph{enumi}.}
\tightlist
\item
  Motor vehicle exceeding 10 tons gross vehicle weight;
\item
  Bus with seating capacity more than 25 including driver;
\end{enumerate}

\hypertarget{medium-vehicles}{%
\subsection{Medium Vehicles}\label{medium-vehicles}}

\begin{enumerate}
\def\labelenumi{\alph{enumi}.}
\tightlist
\item
  Motor vehicle exceeding 3 tons gross vehicle weight but not exceeding 10 tons gross vehicle weight;
\item
  Vehicle with 13-24 passenger seats;
\item
  Tractors above 20 horsepower (hp);
\end{enumerate}

\hypertarget{equipment}{%
\subsection{Equipment}\label{equipment}}

\begin{enumerate}
\def\labelenumi{\alph{enumi}.}
\tightlist
\item
  Road Roller;
\item
  Bull-dozer;
\item
  Mobile crane;
\item
  Pay loaders/excavators
\item
  Road Pavers
\item
  Any other earth-moving and construction equipment
\end{enumerate}

\hypertarget{light-vehicles}{%
\subsection{Light Vehicles}\label{light-vehicles}}

\begin{enumerate}
\def\labelenumi{\alph{enumi}.}
\tightlist
\item
  Any vehicle above 796 CC or weighing not more than 3 tons gross vehicle weight and not exceeding 12 seats (including driver);
\item
  Power-tiller;
\item
  Tractor below 20 horsepower (hp);
\end{enumerate}

\hypertarget{two-wheelers}{%
\subsection{Two Wheelers}\label{two-wheelers}}

\hypertarget{exemption-to-register}{%
\section{Exemption to register}\label{exemption-to-register}}

A motor vehicle belonging to following are exempted from the requirement to register with the Authority;

\begin{enumerate}
\def\labelenumi{\alph{enumi}.}
\tightlist
\item
  Royal Bhutan Army;
\item
  Royal Body Guard; and
\item
  Royal Bhutan Police.
\end{enumerate}

\hypertarget{requirement-for-initial-registration-of-a-motor-vehicle}{%
\section{Requirement for initial registration of a motor vehicle}\label{requirement-for-initial-registration-of-a-motor-vehicle}}

A person wishing to register a motor vehicle must;

\begin{enumerate}
\def\labelenumi{\alph{enumi}.}
\tightlist
\item
  Produce documents in original such as Identity Card for Bhutanese Nationals and Passport/recognized identification documents for Non-Bhutanese as proof of identity, age and eligibility.
\item
  Be present in person or send a representative with a written authorization and the vehicle for registration
\item
  Furnish documentary proof of taking delivery from the authorized dealer/agent/distributor in the country, including invoice in original or copy certified by Department of Revenue \&Customs (DRC), emission standard compliance certificate and evidence to demonstrate that all duties and clearance charges and fees have been paid;
\item
  In the case of a vehicle purchased from other countries by the owner directly without involving local dealer(s), documentary evidence of sale by the authorized dealer and evidence to demonstrate that applicable duties and charges have been paid in Bhutan.
\item
  Produce trade license, certificate of incorporation for registration in name of registered company; certificate of registration with CSO for NGOs or certificate of registration with Commission for Religious Organization/Letter from Dratshang Lhentshog
\item
  Incase of a vehicle on ``allotment'' to entitled government employees by the Ministry of Finance, a letter to this effect, along with a copy of the import license; Tax clearance certificate
\item
  Complete and sign the application (FORM 1) and undertaking (FORM 2) as prescribed; and
\item
  Pay the fee specified in Schedule-1 of the Regulations.
\item
  An application for registration must be completed within 15days of taking delivery of a vehicle from the customs Authority or the authorized dealer, and the vehicle registered with the Authority within this period.
\item
  Ensure that the motor vehicle is four-stroke engine in case it is a two wheeler.
\item
  Ensure that the motor vehicle is at least 796 cubic capacity (cc)
\item
  Ensure that the windshield and front door window glasses of the vehicle are not tinted dark to obstruct visibility.
\item
  Registration of a vehicle damaged while on transit and auctioned by an insurance company may be accepted upon proper certification by a competent workshop and on physical inspection by the Authority.
\item
  The authority shall refuse to register a motor vehicle if the above requirements are not fulfilled.
\end{enumerate}

\textbf{Offence}: Failure to register motor vehicle within 15 days
\textbf{Penalty}: 2 units for every day delayed shall be applied if the vehicle is not registered within 15 days from the date of taking delivery.

\hypertarget{application-for-vehicle-registration}{%
\section{Application for vehicle registration}\label{application-for-vehicle-registration}}

\begin{enumerate}
\def\labelenumi{\alph{enumi}.}
\tightlist
\item
  An applicant for registration of a vehicle must produce the vehicle for physical inspection by an authorized person, at the time of registration.\\
\item
  Only the owner of a vehicle can apply for registration of a vehicle in his name.
\item
  If more than one person owns a vehicle, an application for registration must be made by one of the owners nominated by the other owner (s).
\item
  Despite sub-section (1.8.1) and (1.8.2), the Authority must not register a vehicle in the name of a non-Bhutanese for commercial purposes , except where the vehicle is meant for his/her personal use, if;

  \begin{itemize}
  \tightlist
  \item
    A person has employment in Bhutan;
  \item
    A person has valid work permit;
  \item
    A person resides in the country for more than 6 months;
  \item
    Has valid trader card issued by Ministry of Economic Affairs
  \end{itemize}
\item
  Upon submission of an application by the owner of a vehicle, if the Authority is satisfied that all requirements have been complied with, it shall:

  \begin{itemize}
  \tightlist
  \item
    assign a separate registration number to the vehicle.
  \item
    Supply a certificate of registration to the registered owner; and
  \item
    enter the particulars of the applicant as the registered owner of the vehicle;
  \item
    the particulars of the vehicle;
  \item
    the registration expiry date; and
  \item
    any other information the Authority considers necessary, in its records; and
  \item
    collect fee as set out in schedule I.
  \end{itemize}
\end{enumerate}

\hypertarget{validity-of-registration-certificate-for-different-category-of-vehicle-shall-be}{%
\section{Validity of Registration Certificate for different category of vehicle shall be:}\label{validity-of-registration-certificate-for-different-category-of-vehicle-shall-be}}

\begin{enumerate}
\def\labelenumi{\alph{enumi}.}
\tightlist
\item
  six months to one year for commercial vehicles (Trucks and Taxis);
\item
  validity of registration for a power tiller shall be for Ten years. Yearly extension shall be given subject to road worthiness inspection after the 10th year on annual basis;
\item
  one to five years for all other categories subject to realization of equivalent amount of fees as per schedule-I;
\end{enumerate}

\hypertarget{vehicle-registration-certificate}{%
\section{Vehicle Registration Certificate}\label{vehicle-registration-certificate}}

\begin{enumerate}
\def\labelenumi{\alph{enumi}.}
\item
  a Registration Certificate of a vehicle must contain sufficient information concerning;

  \begin{itemize}
  \tightlist
  \item
    the identity of the vehicle;
  \item
    its registered owner;
  \item
    the period of validity of its registration;
  \item
    the seating capacity (including driver);
  \item
    Gross Vehicle Weight (loading capacity), and
  \item
    any other information the Authority may consider necessary.
  \end{itemize}
\item
  a valid RC must be carried at all times with the vehicle while driving;
\item
  digital possession of the vehicle documents shall be accepted upon verification by the authorized official or authorized police official;
\item
  registration of new vehicles shall be prohibited if an applicant has motor vehicles outstanding dues.
\end{enumerate}

\textbf{Offence}: Failing to possess valid registration certificate
\textbf{Penalty}: 15 units

\begin{enumerate}
\def\labelenumi{\alph{enumi}.}
\setcounter{enumi}{4}
\tightlist
\item
  For the purpose of assigning seating capacity for a vehicle, the authority shall abide by the provision in schedule-II.
\end{enumerate}

\textbf{Offence}: Excess Passenger and Excess load\\
\textbf{Penalty}: 50 units for each person in excess of the permissible number, and 30 units for every excess ton of overload.

\hypertarget{replacement-of-registration-certificate}{%
\section{Replacement of Registration Certificate}\label{replacement-of-registration-certificate}}

\begin{enumerate}
\def\labelenumi{\alph{enumi}.}
\tightlist
\item
  The Authority may issue a replacement of the registration certificate to the registered owner of a vehicle if he/she:

  \begin{itemize}
  \tightlist
  \item
    can prove that the RC has been lost, damaged or destroyed as verified by the Authority or Royal Bhutan Police;
  \item
    surrenders the original RC, if damaged or fully exhausted;
  \item
    pays the applicable fee in schedule 1.
  \end{itemize}
\item
  the Authority must issue a replacement of the registration certificate to the registered owner of a vehicle, without payment of any fee if the registration certificate is fully used.
\item
  in cases of application for replacement of RC where the original documents found to be seized by Traffic Police, the registered owner shall be levied penalty for false declaration.
\end{enumerate}

\textbf{Offence}: False declaration and claim of seized documents\\
\textbf{Penalty}: 40 units

\hypertarget{cd-registration}{%
\section{CD registration}\label{cd-registration}}

\begin{enumerate}
\def\labelenumi{\alph{enumi}.}
\tightlist
\item
  A motor vehicle belonging to the following are entitled for CD registration:

  \begin{itemize}
  \tightlist
  \item
    an Embassy, a High Commission, Consulate or Diplomatic Mission, and its employees holding diplomatic passport;
  \item
    an agency of the United Nations system and its employees holding diplomatic passport, shall have CD number plates and therefore are exempt from paying registration or annual renewal fee.
  \item
    notwithstanding section (14), the Authority shall allot only one CD number or exempt only one vehicle in case of the same person, referred under section (14) owning two or more vehicles from payment of registration fee and other charges.
  \item
    For the purpose of the section 14, Ministry of Foreign Affairs, Royal Government of Bhutan, shall be the authority to certify an organization or an individual as enjoying diplomatic immunity.
  \end{itemize}
\end{enumerate}

\hypertarget{exemption-from-payment-of-registration-and-annual-renewal-fee}{%
\section{Exemption from payment of registration and annual renewal fee}\label{exemption-from-payment-of-registration-and-annual-renewal-fee}}

\begin{enumerate}
\def\labelenumi{\alph{enumi}.}
\tightlist
\item
  A motor vehicle is exempt from paying initial registration or annual renewal fee if it:
\end{enumerate}

\begin{itemize}
\tightlist
\item
  is CD entitled vehicles as specified in section 1.13;
\item
  belongs to the Royal Family and the vehicle allotted a ``BHT`` registration number;
\item
  belongs to a project under special arrangement between the Government and the donor agency, for which a copy of the signed project documents and tax exemption certificate from Ministry of Finance must be made available;
\item
  notwithstanding section (15), a motor vehicle falling under category (15. a \& c) must pay the cost of registration certificate.
\item
  a motor vehicle falling under category (15 c) must start paying the annual renewal fee after the vehicle is handed over to the Government or sold.
\end{itemize}

\hypertarget{vehicle-identification-number-for-registration}{%
\section{Vehicle Identification number for registration}\label{vehicle-identification-number-for-registration}}

\begin{enumerate}
\def\labelenumi{\alph{enumi}.}
\item
  the registered owner of a vehicle shall make sure that the registration number issued by the Authority for the vehicle is fixed to it in accordance with these Regulations.
\item
  the authority while registering shall record vehicle identification numbers (Chassis number) including engine number or any other prescribed identification number clearly in the system and reflect in the RC.
\item
  A person in charge of a registered vehicle must allow an authorized person who has reasonable cause to do so, to inspect the vehicle identification number of the vehicle at any time. If the registered vehicle owner fails to allow authorized person to inspect the vehicle at any time, the owner shall be levied penalty for failing to obey traffic instructions.
  \textbf{Offence}: Failure to obey traffic instructions
  \textbf{Penalty}: 15 Units
\item
  A vehicle imported into Bhutan must display the temporary registration/identification number assigned to it by the authorized dealer or the Customs Authority. The temporary number must be replaced by a permanent registration number within 15 days of taking delivery of the vehicle from the Customs Authority or the authorized dealer.
  1.14.5 The Customs Authority, in case of a vehicle of third country origin, may also provide appropriate insurance coverage, including transit insurance for the vehicle (and charge the cost to the owner) to cover the period that the temporary number exist, being not more than 15 days. Failure to register a motor vehicle within 15 days will be liable for late fee of 2 units per day for every day delayed after 15 days.
  1.14.6 The authority shall not allow a person to drive a motor vehicle which is not registered in the country within the specified time period in these regulations
  Offence: Driving foreign registered vehicle
  Penalty: 30 Units for use of unregistered vehicles.
  1.15 Method of displaying registration number plate\\
  In accordance with section (1.14), two number plates must be fixed, one to the front and one to the rear of the motor vehicle. The size of the number plate, letters and colour scheme must conform to the standard prescribed by the Authority under schedule-III of the Regulations which may be subject to change from time to time.
  1.15.1 A person who uses a motor vehicle on road is guilty of an offence if:
  1.15.1.1 the vehicle does not have number plates fixed to it in accordance with this regulations section
  1.15.1.2 the number on the number plate fixed to the vehicle is wholly or partly obscured and cannot be seen from 20 meters away.
  Offence: Number plate not in line with standards
  Penalty: 20 units
  1.15.1.3 The authority shall require the elected members of parliament to remove the parliament logo after the completion of their tenure within 30 days.
  1.15.1.4 Registration plates of vehicles belonging to the members of parliament shall be allowed to display the parliament logo approved by the Government and shall remove the logo within 30 days after completion of their term.
  Offence: Failure to remove Logo
  Penalty: 50 Units
  1.15.1.5 If a motor vehicle is towing another vehicle, the registration number of a towing vehicle must be fixed to the towed vehicle and must be seen clearly from 20 meters away.\\
  Offence: Number plates not fixed on the towed vehicle
  Penalty: 20 units
  1.16 Renewal of Registration
  The owner of a motor vehicle must arrange to renew the registration certificate on or before the expiry of the current registration:
  1.16.1. On payment of fee in schedule 1; and
  1.16.2. After furnishing the registration certificate to the Authority for amendment.
  Penalty: Nu.10 per day after the expiry of the current registration up-to a maximum of Nu.3000.The renewal starts from the date of expiry of current registration.
  1.16.3. A vehicle which has not renewed the documents for two consecutive years shall be automatically de-registered after issuance of notice to the registered owner through system notification
  1.16.4. If a motor vehicle belongs to project under the special arrangement between the Government and the donor agency, the letter of ownership from the concerned agency should be produced during the time of renewal
  1.16.5. The registration certificate shall stand renewed from the date it fell due, if the annual renewal fee and penalty, where applicable, have been paid.
  1.16.6. The Authority, in case of a commercial vehicle undergoing major repair, may exempt the owner from payment of the late renewal fee or tax for the duration a vehicle remains off-road, provided the owner:
  1.16.6.1 informs the Authority in writing at the time of taking the vehicle for repair;
  1.16.6.2 surrenders the registration certificate to the Authority; and
  1.16.6.3 Furnishes authentic evidence or a certificate from the concerned workshop after the repairs have been completed, showing the duration of repair.
  1.16.6.4 Formula to be used for the purpose of sub-section (1.16.6) shall be:
  (Tax reduction = motor vehicle tax X (number of days spent on repair)
  365 days\\
  1.16.7 Notwithstanding section 1.16, the late penalty for renewal of Registration certificate shall be waived off if a person produces sufficient documentary evidence to show that the registered owner:
  1.16.7.1 was out of the country for which acceptable supporting documents must be provided,
  1.16.7.2 was undergoing medical treatment within or outside the country for which acceptable documents must be provided
  1.16.7.3 the vehicle was subject to court case
  1.16.7.4 was in the prison
  1.17 Road Worthiness Inspection
  The Authority shall inspect a motor vehicle for roadworthiness;
  1.17.1 twice every year for commercial vehicles including taxis;
  1.17.2 once every year for all other category of vehicles.
  1.17.3 All earth moving/construction equipment and power tillers are exempt from fitness inspection requirement.
  1.17.4 A motor vehicle owner shall pay fee for conducting the test of roadworthiness as in schedule-I.
  1.17.5 A person shall not drive a motor vehicle without a valid certificate of roadworthiness.
  1.17.6 Failure to renew roadworthiness certificate on time shall result in payment of penalty of 1 unit per day up to a maximum of 15 units
  1.17.7 A motor vehicle owner failing to renew the certificate of roadworthiness is liable for penalty unless the vehicle owner can provide documentary evidence to prove that he could not produce the vehicle for inspection within the required period because:
  1.17.7.1 he/she was ill;
  1.17.7.2 he/she was out of the country;
  1.17.7.3 he/she was awaiting a court decision; or
  1.17.7.4 The vehicle was off road or in workshop undergoing repair, in which case the owner must inform the Authority in writing in advance along with registration Certificate and furnish authentic evidence or certificate from the workshop after the repair has been completed, showing the duration of repair.
  1.18 Motor Vehicle De-registration
  1.18.1 The Authority shall notify the registered vehicle owner through SMS to the registered mobile number about the failure to renew motor vehicle documents for 1year and six months and the impending deregistration that it entails. Subsequently, the Authority may process automatic de-registration if the vehicle was not renewed for two consecutive years.
  1.18.2 Notwithstanding section 1.18, revival of registration of de-registered vehicle which is proposed to be driven on the road shall be permitted upon providing satisfactory justification and on payment of renewal fees up-to-date together with the penalty \citet{Nu.10} per day for the duration it has not been renewed. Maximum penalty limit of Nu.3000 shall not apply in such cases.
  1.18.3 The formal deregistration of a vehicle which has been scrapped and no more on the road, but its registration cancelled by the system for failing to pay renewal fees for at least two years, shall be permitted upon payment of renewal fees up-to-date with a lump sum penalty of Nu.3000.
  1.19 Suspension of registration of a vehicle being un-roadworthy
  1.19.1 The Authority must suspend the registration of a vehicle if it is satisfied, on reasonable grounds, that the vehicle is unsafe for use on a road.
  1.19.2 The Authority must not withdraw a suspension of registration imposed under sub-regulation (7) until it is satisfied that the defect in the vehicle has been remedied.
  1.19.3 While the registration of a vehicle is suspended under this regulation the vehicle is not allowed to ply on the road.
  Offence: Driving the suspended vehicle
  Penalty: 50 Units
  1.19.4 Notwithstanding section (1.19.3), while the registration of a vehicle is suspended under this regulation, a person may use it to:
  1.19.4.1 take it for repair; or
  1.19.4.2 examine and test the vehicle; or
  1.19.4.3 return it from the repairer; or
  1.19.4.4 take it for inspection by the Authority.
  1.19.5 Suspension under this regulation does not alter the expiry date of registration of a vehicle.
  1.20 Cancellation of motor vehicle registration
  The Authority shall cancel the registration of the vehicle if:
  1.20.1 a registered motor vehicle owner requests the Authority to cancel registration of an un-roadworthy motor vehicle;
  1.20.2 the Authority, upon physical verification, finds the motor vehicle in un-roadworthy condition;
  1.20.3 The authority shall cancel the registration in any other circumstances deemed necessary upon verification
  1.20.4 The owner must produce Central Registry Report prior to cancellation of such un-roadworthy motor vehicles;
  1.21 Alteration and modification of a motor vehicle
  1.21.1 A registered vehicle owner must obtain prior approval for making any alteration to the vehicle
  1.21.2 The authority may approve the vehicle alteration only with the detailed study concerning safety and any other reasons deemed to be considered by the authority.
\end{enumerate}

1.21.3 The owner of a registered vehicle must notify the Authority of the substitution of the vehicle's engine and the engine number of both the former and substitute engine within 15 days of the substitution.

1.21.4 If a registered vehicle is altered (either in construction or appearance) so that the description in registration certificate is different from its altered description, the registered owner of the vehicle must, within 15 days after the alterations are made:
1.21.4.1 present the vehicle to the Authority for inspection by an authorized person;
1.21.4.2 surrender to the authority, the registration certificate of the vehicle.
1.21.4.3 If approved, after inspection, the authority shall make following changes in;
1.21.4.3.1 the registration certificate of the vehicle;
1.21.4.3.2 amend its records accordingly; and
1.21.4.3.3 return the certificate to the registered owner of the vehicle.
1.21.5 The alteration in vehicle construction leading to category change of the vehicle shall not be allowed after its initial registration.
1.21.6 An act of unlawful modifications to a vehicle shall arise if the owner or the driver:
1.21.6.1 Changes color of the vehicle without informing the registering authority
1.21.6.2 Use tinted glasses on any of the car windows, tinted lighting system or window curtains in a vehicle
1.21.6.3 Make use of lights in a vehicle that are unsafe for other road users
1.21.6.4 Extend the size of the vehicle beyond what is specified by the manufacturer
1.21.6.5 Modifies horn with an intention to produce excessive noise
1.21.6.6 Fits extra lights beyond what is fitted by the manufacturer
1.21.6.7 Fits bulbar in a vehicle
1.21.6.8 Notwithstanding the section (1.21) of these regulations, a person shall be instructed to remove the alteration and shall be levied penalty for unauthorized alteration/modification of vehicle
Offence: Unlawful alteration of vehicles
Penalty: 30 units
1.22 Visiting motor vehicles.
1.22.1 The Authority shall allow a motor vehicle registered in other countries to ply temporarily in Bhutan based on the provisions of the bi-lateral/multilateral/regional transport agreement upon payment of applicable fees.
1.22.2 The Authority shall not allow a motor vehicle registered in other countries to operate commercial transport services or carry paid passengers from one point to another point in Bhutan;
1.22.3 The Authority shall require all the vehicles of other countries visiting or plying in Bhutan to abide by the prevailing traffic and transport laws, rules and regulations of Bhutan;
1.22.4 The Authority shall not allow foreign registered vehicles to be driven by the Bhutanese;
1.22.5 The Authority shall not allow Bhutan registered vehicles to be driven by foreign nationals except those registered in his or her name during their stay in the country;
1.22.6 The Authority shall not allow bike rallies unless under special government approval;
1.22.7 The foreign registered vehicle belonging to managerial and professional foreign employees plying more than three months in Bhutan has to endorse their motor vehicle documents upon payment of applicable fees prescribed in schedule 1;
1.22.8 The foreign registered caravan vehicles shall not be allowed to ply in the country.
Offence: Offence in relation to visiting motor vehicles
Penalty: 50 units

1.22.9 Government vehicles of Government of India are however, exempt from payment of fees.\\
1.23 TRANSFER OF OWNERSHIP
1.23.1 A person selling the vehicle must within 15 days of sale report to the Authority in person and transfer the ownership in the name of the person buying the vehicle.
1.23.2 Failure to transfer ownership within prescribed period of 15 days shall result in penalty payment of 2 units per day up to maximum of Nu.3000
1.23.3 An application for transfer of ownership of a vehicle must be accompanied by:
1.23.3.1 a deed of sale which includes the sale price of the vehicle and the signature of the seller and buyer, signed across a legal stamp by both the parties and a witness from each party;
1.23.3.2 CID copy of both the parties;
1.23.3.3 Recent passport size photograph of buyer;
1.23.3.4 the registration certificate for the vehicle;
1.23.3.5 Central Registry Report for verification of hypothecation status by the Authority;
1.23.3.6 Produce trade license and certificate of incorporation for registration in the name of registered company;
1.23.3.7 the fee and charges for change of ownership, including the property transfer tax indicated in schedule 1.
1.23.3.8 Death Certificate, certified family tree and dully signed no objection letter by the family members in case of transfer of ownership from a deceased registered owner to the property transferee
1.24 Payment of transfer tax
1.24.1 Responsibility for payment of property transfer tax shall lie with the seller, except in case;
1.24.1.1 of a vehicle purchased from the Government on allotment;
1.24.1.2 of vehicle sold by international and diplomatic agencies; or
1.24.1.3 of the seller exempt from the transfer tax, in which case, the buyer shall pay the transfer tax unless he can produce a certificate of exemption from the Ministry of Finance.
1.24.1.4 If, in the opinion of the Authority, the sale value has been understated in the sale deed, it is empowered to carry out revaluation of the vehicle using the formula under sub -- section (1.24.2).
1.24.2 For the purpose of charging 1\% (or as may be amended from time to time) vehicle ownership transfer tax, the Authority shall apply depreciation at the following rates on the initial purchase price (on production of original invoice/bill or similar records maintained by the Authority) or the current market value of a similar type, using the diminishing balance method:

1.24.2.1 @ 10\% per annum for first and second year;
1.24.2.2 @ 15\% per annum for third, fourth, fifth and sixth year; and
1.24.2.3 a motor vehicle shall be valued at 10\% of the initial purchase price (as indicated in the original invoice/bill or similar records maintained by the Authority) or the current purchase price (of a new vehicle) of the same or similar description, after the sixth year.
1.24.3 Where a vehicle is seized and auctioned by the financial institution to recover its loan, the previous owner in whose name the loan is outstanding, will be subject to transfer tax on the actual amount received by the property owner after adjustment of the outstanding loan amount.
1.24.4 In case of a vehicle sold as scrap, the owner shall not be liable for transfer tax on the sale value of the scrap. However, the owner must surrender the registration certificate to the Authority for the cancellation of registration.
1.25 Exemption from payment of property transfer tax.
Payment of property transfer tax is exempt in case of the ownership being transferred:
1.25.1 between immediate family members. For the purpose of this section, immediate family members shall refer to parent, children and spouse;
1.25.2 from grandparents to grandchildren;
1.25.3 from a sibling who is the legal custodian of the inherited property to other biological or legally adopted siblings
1.25.4 returned to the main house after the death of the person;
1.25.5 split as one's share of property under joint ownership;
1.25.6 shared under a divorce settlement agreement as per the Marriage Act of Kingdom of Bhutan;
1.25.7 to registered Religious Organizations and registered public benefit organization registered under Chhoedey Lhentshog and Civil Society Organizations respectively;
1.25.8 between exempt international organization;
1.25.9 from children to the parents;
1.25.10 to Zhung Dratshang or Dratshang Lhentshog; and
1.25.11 between members registered under the same census record.
1.26 Refusal to transfer ownership
1.26.1 The Authority may refuse to transfer ownership of a motor vehicle if the requirements under these regulations have not been met.
1.27 Sale of vehicle to a person outside Bhutan
1.27.1 The registered owner of a vehicle, on selling his vehicle to a person residing outside Bhutan must, within 15 days, surrender the registration certificate to the Authority and obtain a no-objection-certificate from the Authority to that effect on payment of fee in schedule I.
1.27.2 Issue of no-objection-certificate in case of a vehicle imported from third country shall be subject to a clearance certificate issued by the Ministry of Finance except for the vehicle scraps.
1.28 Conversion of registration number
1.28.1 The Authority must charge fee in schedule-I for conversion of registration number from government to private or vice versa or taxi to private vehicle.
1.28.2 Notwithstanding sub-section (1.28), conversion of registration number from government to private or vice versa must be subject to a clearance from the Ministry of Finance.
1.28.3 The Authority shall not charge fee for conversion of registration number from taxi to private, upon completion of prescribed life of taxi.
1.29 Restriction on purchase and sale of imported, second hand or reconditioned vehicle
1.29.1 The Authority shall neither permit nor register a second hand/reconditioned vehicle imported into the country, regardless of the country of origin.
1.29.2 Notwithstanding sub-section (1.29):
1.29.2.1 An expatriate may bring a used vehicle on posting in Bhutan but must be re-exported upon completion of term of employment in the country.
1.29.3 For all other matters concerning import, transfer or re-export of a motor vehicle of third country origin the provision of the Bhutan Foreign Vehicles Allotment Rules1994 shall apply.
1.30 Sale and Transfer of ownership of duty and tax free vehicles imported by privileged personnel
In accordance to the Rules on the Sale Tax, Customs and Excise Act of the Kingdom of Bhutan 2000 of Department of Revenue and Customs, the following sections of provisions shall apply in the case of sale and transfer of ownership of Duty and Tax free vehicles imported by the privileged personnel
1.30.1 Sale and Transfer of Duty Exempt Vehicles
1.30.1.1 Transfer of ownership of vehicles (by sale or gift) imported or acquired without payment of customs duty and taxes to non-privileged shall be subject to payment of Customs duty, taxes and charges.
1.30.1.2 Transfer of ownership of vehicle shall be effected only on payment of duty, taxes and other charges to the Department. No new registration of new vehicles or transfer of ownership of any vehicles shall be effected by the Authority without the clearance from the Department.
1.30.1.3 The buyer shall be liable to pay the Customs and Sales tax.
1.30.1.4 No customs duty and Sales Tax shall be levied if a vehicle is sold after a period of five years from the date of its registration.
1.30.2 Disposal of tax/duty exempt vehicles by officials of Diplomatic Mission, International Organizations and Experts on Compilation of their Assignment in Bhutan
1.30.2.1 They can re-export their imported vehicles on completion of their assignments in Bhutan
1.30.2.2 It can be sold on duty and tax exempt basis to another privileged person serving in Bhutan with clearance from the Department of Revenue and Customs provided the buyer has not already imported or placed order for importation a vehicle.
1.30.2.3 It can be sold on duty and tax exempt basis to a Bhutanese official holding vehicle allotment quota.
1.30.2.4 It can be sold in the open market subject to fulfilment of the conditions laid down in 1.30.1.
1.30.2.5 Used vehicles brought on their posting to Bhutan shall not be permitted to be sold or transferred to any Bhutanese or privileged persons by sale or gift and shall have to be re-exported on repatriation. Such vehicles if left in Bhutan shall be confiscated.
1.30.2.6 Where vehicles other than the used vehicles mentioned in Rule (e) above are disposed of in Bhutan, transfer of ownership whether by sale or gift must be completed prior to the departure of the owner
1.30.3 Disposal of official Vehicles of Diplomatic Missions and International Organizations.
1.30.3.1 Vehicles of the Diplomatic Missions and International organizations, if purchased by a non-privileged or is not a Government organization authorized by the Ministry of Finance shall be governed as per the rules laid out in and mentioned in Rules 1.30 above.\\
1.31 Motor Vehicular Emission and Noise
1.31.1 The Authority shall conduct or authorize any qualified person or firms to conduct motor vehicular emission test.
1.31.2 A person shall not use or allow a motor vehicle to be used on a road, if the vehicle is emitting excessive exhaust smoke.
Offence: Emitting excessive smoke
Penalty: 20 units
1.31.3 A commercial vehicle must undergo a test of emission once every six months and all other categories once every year.
Offence: Failure to undergo emission test
Penalty: 30 units
1.31.4 The authorized person or police personnel may require a motor vehicle to undergo emission test if a vehicle is emitting excessive smoke while being driven on a highway.
Offence: Non-compliance to instructions to rectify emission level
Penalty: 30 units if the defects are not rectified and the vehicle tested for emission levels within 30 days.
1.32 Notice prohibiting the use of Motor Vehicles with unacceptable level of emission
1.32.1 The Authority may serve a notice prohibiting the use of a motor vehicle on a road if the emission test carried out shows an unacceptable level of emission from its exhaust.
1.32.2 The owner of a motor vehicle which has been issued a notice under sub-section(1.32.1) shall not use or allow his vehicle to be used on a road until the defects are remedied and the level of emission brought down to an acceptable level.
Offence: Non-compliance to instructions to rectify emission level
Penalty: 30 units for the first offence; 50 units for second offence and grounding of the vehicle thereafter.
1.33 Emission Standard for Motor Vehicle
1.33.1 The Authority shall deploy appropriate mechanical device for the purpose of determining the acceptable level of smoke and shall be subject to review/change from time to time.
1.33.2 Permissible level of emission for every motor vehicle shall be:
1.33.2.1 Diesel: 75\% Hartridge Smoke Unit (HSU) for vehicles registered before 1/1/2005 and 70\% HSU for vehicles registered after 1/1/2005
1.33.2.2 Petrol: 4.5\%Carbon Monoxide(CO) for vehicles registered before 1/1/2005 and 4\%CO for vehicles registered after 1/1/2005
1.33.3 The permissible smoke density level and emission level for all other pollutants present in vehicular exhaust (like hydro carbons, oxides of nitrogen, etc.) shall be as laid down by the National Environment Commission from time to time.
1.33.4 The Authority or authorized person/firm shall charge fee for conducting the test of emission as provided in schedule--I, and the fee may be subjected to review and revision from time to time.
1.34 Pollution under Control Certificate (PUC)
1.34.1 Every motor vehicle must obtain a ``Pollution under Control'' certificate and `sticker' showing that the smoke emission level from that vehicle is within the prescribed limit. The sticker shall be displayed on the upper left corner of the wind screen of the vehicle at all times.
Offence: Failure to produce Emission compliance certificate (PUC)
Penalty: 15 units

1.34.2 A `Pollution under Control' Certificate and sticker can be issued only by the Authority or authorized person/firm to test smoke emission level. Any person found using or carrying a forged Emission sticker and Pollution under Control Certificate, will be penalized as per decisions of the Transport Disciplinary Committees or forwarded to the nearest police station.
1.34.3 In the event of the driver or the owner of the vehicle failing to produce the vehicle for re-test within the time specified or the vehicle on re-test showing an unacceptable level of smoke, the Authority, may suspend the certificate of registration of the vehicle as per Section -23 of the Road Safety \& Transport Act-1999 until the prescribed level is reached.
1.34.4 The certificate remains valid for a period of six months in case of commercial vehicles and one year for other categories of vehicles, from the date of issue of the `pollution under Control' certificate.
1.34.5 The owner or the driver of Emission Failed Motor Vehicle will be issued a notice to rectify the defects within 30 days, and for which the Authority or authorized person/firm shall be permitted to charge fee at the rate prescribed in Schedule-I.
1.34.6 ``Pollution Under Control'' certificates must be renewed on or before expiry of the validity of the current certificate.
1.34.7 The authorized person/firm shall submit monthly reports to the authority using the online reporting system.
1.35 Exemptions from requirement to undergo vehicular emission test
The following motor vehicles are exempt from requiring to undergo vehicular emission test:
1.35.1 Power Tillers;
1.35.2 Earth moving and construction equipment, Antique vehicles kept for show only and not plying on public road.
1.35.3 All new vehicles for the initial period of three years counting from the date of initial registration
1.35.4 The Authority shall issue a certificate of exemption from requirement for emission test for the said duration in sub-section 1.35.3.
1.36 Excessive Noise
A person must not use or allow a motor vehicle with an internal combustion engine to be used on a road, unless the vehicle has a silencing device which:
1.36.1 is securely fixed to its engine so that all the exhaust gases from the engine pass through the silencing device and prevent undue noise; and
1.36.2 does not have attached to it a device capable of producing an open exhaust.
1.36.3 A person must not use or allow a motor vehicle to be used on a road if the vehicle is creating undue noise.
1.36.4 Use of vacuum horns shall not be permitted in vehicles.
1.36.5 Excess 75db of noise level
Offence: Use of vacuum horns
Penalty: 30 units
1.37 Prohibition of Washing of vehicles.
Washing of vehicles in road side brooks, rivers and streams shall be prohibited.
Offence: Washing vehicles in road side brooks, rivers and streams
Penalty: 20 units

\hypertarget{methods}{%
\chapter{Methods}\label{methods}}

We describe our methods in this chapter.

\hypertarget{applications}{%
\chapter{Applications}\label{applications}}

Some \emph{significant} applications are demonstrated in this chapter.

\hypertarget{example-one}{%
\section{Example one}\label{example-one}}

\hypertarget{example-two}{%
\section{Example two}\label{example-two}}

\hypertarget{final-words}{%
\chapter{Final Words}\label{final-words}}

We have finished a nice book.

  \bibliography{book.bib,packages.bib}

\end{document}
